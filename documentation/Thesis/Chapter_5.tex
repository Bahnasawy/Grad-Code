\chapter{Implementation}

\section{General Guidelines}

\textbf{Students can choose between}

\begin{enumerate}
    \item Illustrate your implementation parts
    \item  (Application oriented) “Deliverable” from Selected Approach
    \item (Comparing Algorithm) Experiment Design
\end{enumerate}

The Implementation section is similar to the Specification and Design section in that it
describes the system, but it does so at a finer level of detail, down to the code level. This section is about the realization of the concepts and ideas developed earlier. It can also describe any problems that may have arisen during implementation and how you dealt with them.
Do not attempt to describe all the code in the system, and do not include large pieces of code in this section. Complete source code should be provided separately. Instead pick out and describe just the pieces of code which, for example:

\begin{itemize}
    \item Are especially critical to the operation of the system;
    \item You feel might be of particular interest to the reader for some reason;
    \item Illustrate a non-standard or innovative way of implementing an algorithm, data structure, etc..
    \item You should also mention any unforeseen problems you encountered when implementing the system and how and to what extent you overcame them. Common problems are:
    \item  Difficulties involving existing software, because of, e.g.,  its complexity, lack of documentation;  lack of suitable supporting software;
\end{itemize}
A seemingly disproportionate amount of project time can be taken up in dealing with such
problems. The Implementation section gives you the opportunity to show where that time has
gone

