\chapter{Results and Evaluation}

%Each object has a normal functionality rather than interaction, the user has to decide when he wants to use object for interaction and when he wants to use it naturally. 


\section{General Rules}
In this section you should describe to what extent you achieved your goals.
You should describe how you demonstrated that the system works as intended (or not, as the
case may be). Include comprehensible summaries of the results of all critical tests that were
carried out. You might not have had the time to carry out any full rigorous tests – you may
not even got as far as producing a testable system. However, you should try to indicate how
confident you are about whatever you have produced, and also suggest what tests would be
required to gain further confidence.
This is also the place to describe the reasoning behind the tests to evaluate your results, what
tests to execute, what the results show and why to execute these tests. It may also contain a
discussion of how you are designing your experiments to verify the hypothesis of a more
scientifically oriented project. E.g., describe how you compare the performance of your
algorithm to other algorithms to indicate better performance and why this is a sound
approach. Then summarize the results of the tests or experiments.

You must also critically evaluate your results in the light of these tests, describing its
strengths and weaknesses. Ideas for improving it can be carried over into the Future Work
section. Remember: no project is perfect, and even a project that has failed to deliver what
was intended can achieve a good pass mark, if it is clear that you have learned from the
mistakes and difficulties.
This section also gives you an opportunity to present a critical appraisal of the project as a
whole. This could include, for example, whether the methodology you have chosen and the
programming language used were appropriate


