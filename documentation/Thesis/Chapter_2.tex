\chapter{Background}

\section{Project Context}
The project is intended to be used by linguistic professionals to extract stylometric features and analyze written text to differentiate between authors.

\begin{itemize}
    \item NLP: Natural Language Processing is a subfield of linguistics that is concerned with providing means for a machine to comprehend and analyze human language.
    \item Regular Expression: It is a simple tool used to find patterns in written text.
    \item Literary style: It is an approach to writing that is unique to each writer through the use of different vocabulary, grammar, techniques, etc..
\end{itemize}

\section{Existing Solutions}


\section{General Guidelines}
The purpose of the Background section is to provide the typical reader with information that they cannot be expected to know, but which they will need to know in order to fully
understand and appreciate the rest of the report
It should explain why the project is addressing the problem described in the
report, indicate an awareness of other work relevant to this problem and show clearly that the
problem has not been solved by anyone else. This section may describe such things as:


\begin{itemize}
    \item The wider context of the project;
    \item The problem that has been identified
    \item Likely stakeholders within the problem area
    \item Any theory associated with the problem area
    \item Any constraints on the approach to be adopted
    \item Existing solutions relevant to the problem area, and why these are unsuitable or Insufficient in this particular case Methods and tools that your solution may be based on or use to solve the problem and so on.
\end{itemize}
The wider context of the project includes such things as its non-computing aspects. So, for
example, if you are producing software or any other products, including business recommendations, for a specific organization then you should describe aspects of that organizations business that are relevant to the project.

